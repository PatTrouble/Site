<HTML>
	<HEAD>
	<style>
	table{
		border-collapse: collapse;;
		border:1;
	}
	li {
		padding-bottom: 10;
	}
	td,th {
		padding-top: 20;
		padding-bottom: 20;
		padding-left: 20;
		padding-right: 20;
		text-align: center;	
		valign:middle;		
	}
	</style>
	<script > 
		function toggle(idRep) {
			var ele = document.getElementById(idRep);
			
			if(ele.style.display == "block") {
					ele.style.display = "none";		
			}
			else {
				ele.style.display = "block";
			}
		} 
	</script>
	<!-- https://www.codecogs.com/latex/eqneditor.php -->
	<title>8THE105 - Ensembles, relations et fonctions</title>
	</HEAD>
	<BODY>
	<h1>Laboratoire 1</h1>
	
	<H2>La logique</h2>
	
	<ol>
	<li>Soit les deux propositions suivantes (10 pts):
	<ul>
		<li><img valign=middle src="http://latex.codecogs.com/svg.latex?
		\left(\neg p \wedge \neg q 	\wedge r\right) \vee  
		\left(\neg p \wedge q 		\wedge r\right) \vee
		\left(p 	\wedge \neg q 	\wedge \neg r\right) \vee
		\left(p 	\wedge \neg q 	\wedge r\right)"/></li>
		<li><img valign=middle src="http://latex.codecogs.com/svg.latex?
		\neg\left(\left(p \vee \neg r\right) \wedge \left(\neg p \vee q\right)\right)"/></li>
	</ul>                                                 
	
	<p>Vérifier que ces deux propositions sont équivalentes avec une table de vérité.  Si elles le sont démontrer qu'elles sont équivalentes en utilisant les propriétés vues en classe.  À chaque étape, vous devez écrire le nom de la propriété utilisée (celles qui ont un nom).</p></li>
	
	<li>Considérer les énoncés suivant (15 pts):
	<ul>
		<li>Tous les produits Apple sont dispendieux.</li>
		<li>Certains produit dispendieux sont de bonne qualité.</li>
		<li>Les produits Apple sont de bonne qualité.</li>
	</ul>

	<p>Utiliser uniquement les quantificateurs universels 
	<img valign=middle src="http://latex.codecogs.com/svg.latex?(\forall)"/> et existentiels <img valign=middle src="http://latex.codecogs.com/svg.latex?(\exists)"/> de même que des conjonctions 
	<img valign=middle src="http://latex.codecogs.com/svg.latex?(\wedge)"/>,  des disjonctions <img valign=middle src="http://latex.codecogs.com/svg.latex?(\vee)"/> et des négations <img valign=middle src="http://latex.codecogs.com/svg.latex?(\neg)"/> afin de symboliser ces trois énoncés.  Clarifier explicitement les fonctions propositionnelles à utiliser.</p></li>
	
	
<!--	<li>Représenter graphiquement par un diagramme de Venn la zone représentant les expressions suivantes.  Attention à lisibilité. (10 pts)
	<ul>
		<li><img valign=middle src="http://latex.codecogs.com/svg.latex?\overline{A}\cup B \cap \overline{C}"/></li>
		<li><img valign=middle src="http://latex.codecogs.com/svg.latex?A\cap (\overline{B \cup C})"/></li>
	</ul></li>
	
	<li>On a catégorisé les dépenses d’un groupe de 160 étudiants. Parmi ceux-ci, 81 possède un cellulaire, 97 une automobile et 89 un logement. De plus, 21 ont ces trois dépenses et 4 n’en n’ont aucune. Finalement, 140 ont un cellulaire ou une automobile et 151 ont une automobile ou un logement.  Représenter cette situation par un diagramme de Venn et trouver combien d’étudiants ont seulement un cellulaire  (10 pts)</li> -->
</ol>
		

	<h2>Remise</h2>

	<ul>
		<li>5 points sont réservés pour la présentation du travail.</li>		
		<li>Remettre 1 copie par équipe de 2 ou 3 étudiants avant le <b>jeudi 11 septembre avant 19h00</b> dans ma case au 4me étage au DIM ou en main propre.</li>
	</ul>
	</BODY>
</HTML>






